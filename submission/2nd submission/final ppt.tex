
\documentclass{beamer}

\usetheme{Szeged}

\title{Face Detection}

% A subtitle is optional and this may be deleted
\subtitle{A Survey}

\author{Himani Bhatt(1421003) \and Janvi Shah(1421004)\\ \and Priyanka Nimavat(1421010) \and Sahil Desai(1421014}

\institute[IET-Ahmedabad University] % (optional, but mostly needed)
{
  Present To:\\
  Prof.Mehul Raval\\
  
  \and
  Prof.Ratnik Gandhi\\
  
  Dhruv Gupta
  }
\date{}

% This is only inserted into the PDF information catalog. Can be left
% out. 

% If you have a file called "university-logo-filename.xxx", where xxx
% is a graphic format that can be processed by latex or pdflatex,
% resp., then you can add a logo as follows:

% \pgfdeclareimage[height=0.5cm]{university-logo}{university-logo-filename}
% \logo{\pgfuseimage{university-logo}}

% Delete this, if you do not want the table of contents to pop up at
% the beginning of each subsection:

% Let's get started
\begin{document}

\begin{frame}
  \titlepage
\end{frame}

\begin{frame}{Introduction}
  \begin{itemize}
  \item {
    A face detector has to tell whether an image of arbitrary size contains a human face and if so, where it is.
  }
  \item {
    One natural framework for considering this problem is that of binary classification, in which a classifier is constructed to minimize the min classification risk.
  }
  \item{
  The algorithm must minimize both the false negative and false positive rates in order to achieve an acceptable performance.
  }
  \item{
  The challenges associated with face detection are:Pose,Presence or absence of structural component,Facial expression,Occlusion,Image orientation,Imaging conditions
  }
  \end{itemize}
\end{frame}

% You can reveal the parts of a slide one at a time
% with the \pause command:
\begin{frame}{Methods}
Face Detection methods can be classified in two approaches
\begin{itemize}
\item Featured based Approach
\item Image Based Approach
\end{itemize}
\end{frame}

\begin{frame}{Featured based Approach}

\begin{itemize}
\item Edges
\item Gray-Level
\item Color
\item Motion
\item Feature searching
\item Snakes
\item Deformable Templates
\end{itemize}

\end{frame}

\begin{frame}{Image Based Approach}
Neural Networks


\end{frame}

\begin{frame}[label=main]
\frametitle{Viola Jones Face Detection Algorithm}
\begin{itemize}
\item Haar Features:-Haar features are used to detect the presence of particular feature in the given image. Each features resulting a single value which is calculated by
subtracting the sum of pixels under white rectangle from the sum of pixel under black rectangle.

\item Integral Image:-Rectangle features can be computed very rapidly using an intermediate representation for the image which we call the integral image.

\item Adaboost:-Adaboost is machine learning algorithm which helps in finding only the best features among all.

\item Cascading:-Images with multiple faces would suffer from multiple false alarming of face detected. Cascading helps to remove non-face sub-windows to increase detection rate.

\end{itemize}

\end{frame}

\begin{frame}

\frametitle{References}
\begin{thebibliography}{4}

\bibitem{lamport94}
Face Detection,
\emph{A Survey Erik Hjelm°as1}.Department of Informatics, University of Oslo,norway.

\bibitem{lamport94}
Neural Network-Based Face Detection
\emph{Henry A. Rowley,Henry A. Rowley, Takeo Kanade}. Computer Vision and Pattern Recognition1996.

\bibitem{lamport94}
Detecting Faces in Images,\emph{A Survey Ming-Hsuan Yang, Member, IEEE, David J. Kriegman, Senior Member, IEEE, and Narendra Ahuja, Fellow}.IEEE.
 
\bibitem{lamport94}
Robust Real-Time Face Detection,
\emph{PAUL VIOLA}. Microsoft Research, One Microsoft Way, Redmond, WA 98052, USA, July 11, 2003.
 
\bibitem{lamport94}
Rapid Object Detection using a Boosted Cascade of Simple
Features,
\emph{Paul Viola Michael Jones}. Mitsubishi Electric Research Labs Compaq CRL 201 Broadway, 8th FL One Cambridge Center Cambridge, MA 02139 Cambridge, MA 02142.
  
\end{thebibliography}

\end{frame}

% All of the following is optional and typically not needed. 

\end{document}


